\documentclass[letterpaper, 12pt]{article}
\usepackage[spanish, es-tabla]{babel} %%Paquete español para mac
\usepackage[utf8]{inputenc} %% Para unicode
\usepackage{graphicx} %% Para incluir figuras
\usepackage{ifpdf}
\usepackage[table]{xcolor}
\usepackage{xcolor}
\usepackage{ amssymb }
\DeclareGraphicsExtensions{.pdf}
\usepackage{fullpage}
\usepackage{amsmath}
\usepackage[all]{xy}
\setcounter{totalnumber}{5}
\renewcommand{\textfraction}{0.1}
\usepackage[cmex10]{amsmath}
\usepackage{amsmath}
\usepackage{diagbox} % Para la celda con la diagonal
\usepackage{amssymb}
\usepackage{float}
\usepackage{esvect}
\decimalpoint
\usepackage{url}
\graphicspath{imagenes/}
\usepackage{hyperref}
\usepackage{biblatex} 
\usepackage{csquotes}
\addbibresource{Citas.bib}
\hypersetup{colorlinks=false,bookmarksopen=true,linkbordercolor={1 1 1}}
\usepackage{subcaption}
\definecolor{lightyellow}{RGB}{255, 255, 153}

\usepackage{enumerate} 
\setlength{\parindent}{0pt}
\begin{document}
%%%%%%%%%%%%%%%%%%%%%%%%%%
%%%%%%   ENCABEZADO  %%%%%
%%%%%%%%%%%%%%%%%%%%%%%%%%
\vspace*{-1cm}
\includegraphics[width=2.2cm]{logo_usm.png}
\vspace*{-1.8cm}

\hspace*{2cm}
 \begin{tabular}{l}
  {\ Universidad Técnica Federico Santa María}\\
  {\ Departamento de Informática}\\
  {\ INF245 - Arquitectura y Organización de Computadores}\
 \end{tabular}
 \hfill 
\vspace*{5mm}
\begin{center}
{\LARGE\bf Laboratorio 2}\\
\vspace*{2.5mm}
\today\\
Jose Ignacio Valenzuela - Jose.valenzuelavo@usm.cl - 202273635-6\\

Sergio Cárcamo Naranjo - sergio.carcamon@usm.cl - 202273512-0\\
\end{center}
\hrule\vspace*{2pt}\hrule

\setlength{\parindent}{0pt}
% ***********************************************
% ***********************************************
\vspace*{15pt}
\section{Resumen}
En este laboratorio se diseñó un circuito combinacional utilizando \textit{Logisim}, cuyo objetivo es representar en un display de 7 segmentos el tratamiento médico asignado a pacientes en 16 habitaciones diferentes de un hospital. Este problema fue planteado como parte del desafío del Doctor Haus, y busca integrar conocimientos de lógica digital, mapas de Karnaugh y diseño de circuitos.

Cada habitación está identificada por una dirección de 4 bits, lo que permite consultar individualmente a cada paciente. Según el diagnóstico o tratamiento requerido, el circuito entrega un símbolo específico mediante un display de 7 segmentos. El desarrollo de este laboratorio permite profundizar en la implementación de lógica combinacional de manera optimizada.

\section{Entradas, salidas y supuestos}

\textbf{Entradas:} la entrada al sistema de lógica combinacional es un bus de 4 bits que identifica la habitación de la cual se quiere conocer el estado del paciente.\\

\textbf{Salidas:} la salida del sistema de lógica combinacional es un bus de 7 bits que corresponde a combinaciones que se despliegan en un display de 7 segmentos para identificar el estado del paciente que está en la habitación cuyo número se ingresó en la entrada.\\

\textbf{Supuestos:} 
\begin{enumerate}
    \item Estados: el sistema es estático, es decir, el estado de los pacientes en las habitaciones no cambia en el tiempo por lo que solo existe una asignación combinacional entrada-salida.
    \item Validez de la entrada: no se consideraron entradas inválidas, es decir, el sistema siempre espera una entrada de exactamente 4 bits bien definidos (0-1).
    \item Encendido del sistema: se consideró un sistema siempre activo, sin set-reset.
\end{enumerate}

\section{Diccionario de correspondencia}

Tal como se explicó anteriormente, cada habitación tiene un paciente y cada paciente tiene y cada paciente tiene un estado de salud asociado a un binario de 7 bits.

\subsection{Tabla Habitaciones - Pacientes}
A continuación se muestra la tabla de habitaciones y acciones de pacientes según enunciado. En este contexto se asocia el número de habitación a la entrada de 4 bits:
\begin{center}
    \begin{tabular}{|c|c|c|}
    \hline
    Habitación (decimal) & Habitación (binario) & Estado paciente\\

\hline 0 & 0000 & Rayos X \\
\hline 1 & 0001 & Lupus \\
\hline 2 & 0010 & Observación \\
\hline 3 & 0011 & Ecografía \\
\hline 4 & 0100 & Punción lumbar \\
\hline 5 & 0101 & Urgencia \\
\hline 6 & 0110 & Fiebre \\
\hline 7 &0111 & Rayos X \\
\hline 8 & 1000 & Análisis diferencial \\
\hline 9 & 1001 &Biopsia\\
\hline 10 & 1010 & Sarcoidosis \\
\hline 11 & 1011 & Electrocardiograma \\
\hline 12 & 1100 & Ecografía \\
\hline 13 & 1101 & Coronavirus \\
\hline 14 & 1110 & Punción Lumbar \\
\hline 15 & 1111& Observación \\
\hline
    \end{tabular}
\end{center}
\subsubsection{Tabla Estados Pacientes - salida 7 Bits}

Para hacer la tabla de estados pacientes versus salida de 7 bits es necesario hacer una asignación de bits a leds en el display de 7 segmentos.

Típicamente un display de 7 segmentos enumera los LED según la configuración que se muestra a continuación, acompañado del diagrama de conexiones que también se utiliza en \textit{Logisim}.
\begin{center}
\includegraphics[width=4.2cm]{7seg.png}
\end{center}
Al hacer pruebas en \textit{Logisim} es posible notar que cuando en uno de los pines del display hay un 1 lógico, el led asociado a este pin se enciende\footnote{Esto es curioso porque los display físicos reales suelen ser de ánodo común, lo que significa que los led se prenden en el 0 lógico y se apagan en el 1 lógico, pero afortunadamente \textit{Logisim} nos salva de esta contra intuitividad}. Esto se puede observar en la imagen a continuación, donde se hace una prueba encendiendo el led \textit{e} del display conectandolo a una fuente de poder.
\begin{center}
\includegraphics[width=5cm]{prueba_led_logisim.png}
\end{center}
Con esto, y observando el diccionario de asignaciones es posible hacer una tabla de salidas de 7 bits asociadas a las posibles entradas.

A continuación se muestran los valores de salidas de 7 bits asociados a las configuraciones del display para las posibles salidas a mostrar:
\begin{enumerate}
    \item Procedimientos:
    \begin{center}
\includegraphics[width=12cm]{7_seg_procedimeintos.png}
\end{center}
\item Enfermedades:
\begin{center}
\includegraphics[width=12cm]{7_seg_enfermedades.png}
\end{center}
\end{enumerate}

Esto se traduce en una asignación de la siguiente forma:
\begin{center}
    \begin{tabular}{|c|c|c|c|c|c|c|c|}
     \hline
     Enfermedad/Procedimiento&a&b&c&d&e&f&g\\
     \hline
     Punción Lumbar & 1&1&0&0&1&1&1\\
     \hline
     Análisis Diferencial & 1&1&1&0&1&1&1\\
     \hline
     Biopsia & 1&1&1&1&1&1&1\\
     \hline
     Electrocardiograma & 0&1&1&0&1&1&0\\
     \hline
     Rayos X & 0&1&1&0&1&1&1\\
     \hline
     Ecografía & 1&0&0&1&1&1&1\\
     \hline
     Fiebre & 1&0&0&0&1&1&1\\
     \hline
     Observación & 1&1&1&1&1&1&0\\
     \hline
     Urgencia & 0 & 1 & 1&1&1&1&0\\
     \hline
     Lupus & 0&0&0&1&1&1&0\\
     \hline
     Sacroidosis & 1&0&1&1&0&1&1\\
     \hline
     Coronavirus & 1&0&0&1&1&1&0\\
     \hline
    \end{tabular}
\end{center}
Con esta información se puede crear la tabla de verdad general.
\section{Tabla de verdad general}
Combinando la información antes expuesta se llega a la siguiente tabla, donde A B C D son los bits de la entrada del mas significativo al menos significativo.
\begin{center}
\begin{tabular}{|c|c|c|c|c|c|c|c|c|c|c|}
\hline
A&B&C&D&a&b&c&d&e&f&g\\
\hline

\hline 0 & 0 & 0 &  0 & 0 & 1 & 1 & 0 & 1 & 1 & 1 \\

\hline 0 & 0 & 0 & 1 & 1 & 0 & 0 & 0 & 1 & 1  & 0 \\
\hline
0 & 0 & 1 & 0  & 1 & 1 & 1 & 1 & 1 & 1 & 0 \\
\hline0 & 0 & 1 & 1  & 1 & 0 & 0 & 1 & 1 & 1 & 1 \\
\hline0 & 1 & 0 & 0  & 1 & 1 & 0 & 0 & 1 & 1 & 1 \\
\hline0 & 1 & 0 & 1  & 0 & 1 & 1 & 1 & 1 & 1 & 0 \\
\hline 0 & 1 & 1 & 0  & 1 & 0 & 0 & 0 & 1 & 1 & 1 \\
\hline0 & 1 & 1 & 1  & 0 & 1 & 1 & 0 & 1 & 1 & 1 \\
\hline1 & 0 & 0 & 0  & 1 & 1 & 1 & 0 & 1 & 1 & 1 \\
\hline1 & 0 & 0 & 1  & 1 & 1 & 1 & 1 & 1 & 1 & 1 \\
\hline1 & 0 & 1 & 0  & 1 & 0 & 1 & 1 & 0 & 1 & 1 \\
\hline1 & 0 & 1 & 1  & 0 & 1 & 1 & 0 & 1 & 1 & 0 \\
\hline1 & 1 & 0 & 0  & 1 & 0 & 0 & 1 & 1 & 1 & 1 \\
\hline1 & 1 & 0 & 1  & 1 & 0 & 0 & 1 & 1 & 1 & 0 \\
\hline1 & 1 & 1 & 0  & 1 & 1 & 0 & 0 & 1 & 1 & 1 \\
\hline1 & 1 & 1 & 1 & 1 & 1 & 1 & 1 & 1 & 1 & 0\\
\hline

\end{tabular}
\end{center}
\section{Mapas de Karnaugh y Ecuaciones Booleanas por salida}
A continuación se crean los mapas de Karnaugh por bit de salida para calcular cada ecuación booleana asociada. Para todos los Mapas de Karnaugh se usó el orden de Grey y se usaron grupos de unos.
\begin{enumerate}
    \item Mapa de Karnaugh y ecuación segmento a:
\begin{center}
\begin{tabular}{|l|l|l|l|l|}
\hline
       \diagbox{AB}{CD} & 00 & 01 & 11 & 10 \\
\hline
00 & 0  & 0  & \cellcolor{yellow}1 & \cellcolor{yellow}1 \\
\hline
01 & \cellcolor{yellow}1  & 0  & 0  & \cellcolor{yellow}1 \\
\hline
11 & \cellcolor{yellow}1  & \cellcolor{yellow}1  & \cellcolor{yellow}1  & \cellcolor{yellow}1 \\
\hline
10 & \cellcolor{yellow}1  & \cellcolor{yellow}1  & 0  & \cellcolor{yellow}1 \\
\hline
\end{tabular}
\end{center}

La ecuación asociada es:
$$
a = \bar{A} \bar{B} C+C \bar{D}+A \bar{C}+B \bar{C} \bar{D}+A B
$$
    \item Mapa de Karnaugh y ecuación segmento b:
    \begin{center}
\begin{tabular}{|l|l|l|l|l|}
\hline
\diagbox{AB}{CD} & 00 & 01 & 11 & 10 \\
\hline 00 & \cellcolor{yellow}1 & 0 & 0 & \cellcolor{yellow}1 \\
\hline 01 & \cellcolor{yellow}1 & \cellcolor{yellow}1 & \cellcolor{yellow}1 & 0 \\
\hline 11 & 0 & 0 & \cellcolor{yellow}1 & \cellcolor{yellow}1 \\
\hline 10 & \cellcolor{yellow}1 & \cellcolor{yellow}1 & \cellcolor{yellow}1 & 0 \\
\hline
\end{tabular}
    \end{center}
    La ecuación asociada es:
    $$
b = \bar{A} \bar{B} \bar{D}+\bar{A} B \bar{C}+B C D+A B C+A C D+A \bar{B} \bar{C}
    $$
    \item Mapa de Karnaugh y ecuación segmento c:
    \begin{center}
        \begin{tabular}{|c|c|c|c|c|}
        \hline
\diagbox{AB}{CD} & 00 & 01 & 11 & 10 \\
\hline 00 & \cellcolor{yellow}1 & 0 & 0 & \cellcolor{yellow}1 \\
\hline 01 & 0 & \cellcolor{yellow}1 & \cellcolor{yellow}1 & 0 \\
\hline 11 & 0 & 0 & \cellcolor{yellow}1 & 0 \\
\hline 10 & \cellcolor{yellow}1 & \cellcolor{yellow}1 & \cellcolor{yellow}1 & \cellcolor{yellow}1\\
\hline
\end{tabular}
    \end{center}
    La ecuación asociada es:
$$
c = \bar{A} \bar{B} \bar{D}+\bar{A} B D+A C D+A \bar{B}
$$
    \item Mapa de Karnaugh y ecuación segmento d:
    \begin{center}
        \begin{tabular}{|c|c|c|c|c|}
\hline
\diagbox{AB}{CD} & 00 & 01 & 11 & 10 \\
\hline 00 & 0 & \cellcolor{yellow}1 & \cellcolor{yellow}1 & \cellcolor{yellow}1 \\
\hline 01 & 0 & \cellcolor{yellow}1 & 0 & 0 \\
\hline 11 & \cellcolor{yellow}1 & \cellcolor{yellow}1 & \cellcolor{yellow}1 & 0 \\
\hline 10 & 0 & \cellcolor{yellow}1 & 0 & \cellcolor{yellow}1 \\
\hline
\end{tabular}
    \end{center}

    La ecuación asociada es: 
    $$
d = \bar{C} D+\bar{A} \bar{B} C+A \bar{B} C \bar{D}+A B C D+A B \bar{C} \bar{D}
    $$
    \item Mapa de Karnaugh y ecuación segmento e:

    \begin{center}
        \begin{tabular}{|c|c|c|c|c|}
        \hline
\diagbox{AB}{CD} & 00 & 01 & 11 & 10 \\
\hline 00 &\cellcolor{yellow} 1 & \cellcolor{yellow}1 & \cellcolor{yellow}1 & \cellcolor{yellow}1 \\
\hline 01 & \cellcolor{yellow}1 & \cellcolor{yellow}1 & \cellcolor{yellow}1 & \cellcolor{yellow}1 \\
\hline 11 & \cellcolor{yellow}1 & \cellcolor{yellow}1 & \cellcolor{yellow}1 & \cellcolor{yellow}1 \\
\hline 10 & \cellcolor{yellow}1 & \cellcolor{yellow}1 & \cellcolor{yellow}1 & 0 \\
\hline
\end{tabular}

    \end{center}
    La ecuación asociada es:
$$
e = \bar{A}+B+\bar{C}+D
$$
    \item Mapa de Karnaugh y ecuación segmento f:
    \begin{center}
        \begin{tabular}{|l|l|l|l|l|}
\hline
\diagbox{AB}{CD} & 00 & 01 & 11 & 10 \\
\hline 00 & \cellcolor{yellow}1 & \cellcolor{yellow}1 & \cellcolor{yellow}1 & \cellcolor{yellow}1 \\
\hline 01 & \cellcolor{yellow}1 & \cellcolor{yellow}1 & \cellcolor{yellow}1 & \cellcolor{yellow}1 \\
\hline 11 & \cellcolor{yellow}1 & \cellcolor{yellow}1 & \cellcolor{yellow}1 & \cellcolor{yellow}1 \\
\hline 10 & \cellcolor{yellow}1 & \cellcolor{yellow}1 & \cellcolor{yellow}1 & \cellcolor{yellow}1 \\
\hline
\end{tabular}
    \end{center}
    La ecuación asociada es:
    $$
f = A+\bar{A}
    $$
    \item Mapa de Karnaugh y ecuación segmento g:
    \begin{center}
        \begin{tabular}{|c|c|c|c|c|}
        \hline
\diagbox{AB}{CD} & 00 & 01 & 11 & 10 \\
\hline 00 & \cellcolor{yellow}1 & 0 & \cellcolor{yellow}1 & 0 \\
\hline 01 & \cellcolor{yellow}1 & 0 & \cellcolor{yellow}1 & \cellcolor{yellow}1 \\
\hline 11 & \cellcolor{yellow}1 & 0 & 0 & \cellcolor{yellow}1 \\
\hline 10 & \cellcolor{yellow}1 & \cellcolor{yellow}1 & 0 & \cellcolor{yellow}1 \\
\hline
\end{tabular}
    \end{center}
    La ecuación asociada es:
    $$
g = \bar{C}\bar{D} + \bar{A}BC + AC\bar{D}+A\bar{B}\bar{C}D
    $$
\end{enumerate}

\section{Simulaciones en \textit{Logisim}}

A continuación, algunas simulaciones de consultas.
    \item

    \begin{figure}[H]
    \centering
    \includegraphics[width=0.6\textwidth]{0000.png}
    \caption{Paciente 0 (Mark Grayson, Rayos X)}
    \end{figure}

    \begin{figure}[H]
    \centering
    \includegraphics[width=0.6\textwidth]{0100.png}
    \caption{Paciente 4 (Marisa Kirisame, Punción Lumbar)}
    \end{figure}

    \begin{figure}[H]
    \centering
    \includegraphics[width=0.6\textwidth]{1000.png}
    \caption{Paciente 8 (Bruja Lunar Ranni, Análisis Diferencial)}
    \end{figure}

\section{Conclusiones}

Basado en los resultados de este informe, podemos concluir que, en efecto, se puede implementar un sistema de lógica combinacional que permita transmitir información de los pacientes y sus tratamientos, estados y diagnósticos sólo a través de un display de 7 segmentos, por lo que tanto como los primos y el Doctor Haus estarán contentos.

\end{document}
