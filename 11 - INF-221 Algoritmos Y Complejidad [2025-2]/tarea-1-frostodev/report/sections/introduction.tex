El análisis de algoritmos es un pilar fundamental en las Ciencias de la Computación, permitiendo evaluar y predecir el comportamiento de las soluciones de software en términos de eficiencia y consumo de recursos. En particular, el estudio de algoritmos de ordenamiento y multiplicación de matrices representa un área clásica y bien establecida, donde la complejidad teórica es ampliamente conocida y documentada. Sin embargo, la brecha entre el desempeño teórico y el comportamiento práctico de un algoritmo bajo condiciones reales de ejecución (considerando factores como la arquitectura del hardware, gestión de memoria caché y el patrón de acceso a datos) motiva la necesidad de un análisis experimental riguroso.

Este reporte se enmarca en el campo del \textbf{Análisis y Diseño de Algoritmos}, específicamente en la evaluación realista de su desempeño. Mientras que el análisis teórico provee cotas superiores e inferiores de complejidad, el análisis experimental permite validar estas cotas, identificar parámetros ocultos y observar el comportamiento real en escenarios concretos, lo cual es crucial para la selección informada de algoritmos en aplicaciones prácticas. Existen estudios previos que comparan el rendimiento de algoritmos clásicos, pero estos suelen realizarse en contextos controlados y con configuraciones específicas, dejando espacio para explorar su comportamiento en diferentes dominios de datos y tamaños de entrada, particularmente en el contexto más limitado que estudiantes de pregrado como nosotros tenemos.

Tenemos como objetivo realizar un \textbf{análisis experimental} del tiempo de ejecución y uso de memoria de cinco algoritmos de ordenamiento: Insertion Sort, Merge Sort, Quick Sort (con la mediana como pivote), un algoritmo inventado denominado Panda Sort, y el algoritmo de la librería estándar de C++ (`std::sort`) y dos algoritmos de multiplicación de matrices: el método Naive y el algoritmo de Strassen. La \textbf{pregunta central} que guía este análisis es: ¿hasta qué punto el rendimiento práctico de estos algoritmos, medido experimentalmente, se alinea con sus complejidades teóricas esperadas bajo diferentes distribuciones de datos (arreglos ordenados, inversos, aleatorios; matrices densas, dispersas, diagonales) y distintos tamaños de entrada?

El propósito de este reporte es doble: primero proveer una comparación cuantitativa del desempeño de estos algoritmos en una variedad de casos de prueba, generando evidencia empírica que complemente el análisis teórico visto en el curso; y segundo introducir y evaluar el desempeño de Panda Sort, un algoritmo de ordenamiento original desarrollado para esta tarea, cuya lógica y diseño se documentan y analizan. La \textbf{novedad} de este trabajo en el contexto de un curso de pregrado radica en la integración de un algoritmo propio (Panda Sort) dentro de un conjunto de métodos clásicos, permitiendo una comparación directa y en igualdad de condiciones, así como en la aplicación de una metodología de medición sistemática para ambos problemas (ordenamiento y multiplicación de matrices) que considera no solo el tiempo de ejecución sino también el consumo de memoria.