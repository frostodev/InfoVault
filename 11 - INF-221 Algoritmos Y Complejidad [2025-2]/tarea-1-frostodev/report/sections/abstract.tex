Esta investigación realiza un \textbf{Análisis experimental de la complejidad computacional} de algoritmos de ordenamiento y multiplicación de matrices. Se implementaron en C++ cinco algoritmos de ordenamiento (Insertion Sort, Merge Sort, Quick Sort con selección aleatoria de pivote, Panda Sort y Sort de la STL) y dos de multiplicación de matrices (Naive y Strassen), midiéndose su rendimiento en términos de tiempo de ejecución y uso de memoria para distintos tamaños y tipos de entrada. Los resultados se visualizaron mediante gráficos generados en Python, permitiendo contrastar el comportamiento práctico con la complejidad teórica esperada.

Se concluye que, si bien la complejidad teórica predice el crecimiento asintótico, factores como la implementación, el lenguaje y la estructura de los datos introducen variaciones significativas en el desempeño real. Este estudio sienta las bases para futuras investigaciones sobre optimizaciones aplicadas y el análisis de algoritmos en escenarios de gran escala.