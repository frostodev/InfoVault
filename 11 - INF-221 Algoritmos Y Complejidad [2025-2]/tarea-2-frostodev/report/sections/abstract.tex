El presente informe detalla la implementación y análisis comparativo de distintas técnicas algorítmicas para resolver un problema de optimización de productividad en la empresa ficticia CppCorp. El problema consiste en particionar una fila de empleados en equipos contiguos para maximizar la productividad total, la cual se define por reglas específicas de lenguajes de programación y contribuciones individuales.

El objetivo principal es comparar el rendimiento y la calidad de las soluciones obtenidas mediante cuatro enfoques diferentes. Se implementó una solución óptima mediante Fuerza Bruta (Backtracking) que explora todo el espacio de soluciones. Paralelamente, se desarrolló la solución óptima eficiente utilizando Programación Dinámica.

Además, se implementaron dos heurísticas Greedy subóptimas , diseñadas para encontrar soluciones rápidas pero no necesariamente óptimas. El análisis se centra en medir experimentalmente el tiempo de ejecución y el uso de memoria de cada algoritmo frente a un conjunto de casos de prueba con distintos tamaños de entrada.

Los resultados obtenidos permiten cuantificar el trade-off entre la optimalidad de una solución y su eficiencia computacional. Se busca demostrar la inviabilidad de la fuerza bruta para entradas grandes, la eficiencia de la programación dinámica para encontrar el óptimo, y analizar qué tan cercanas al óptimo son las soluciones entregadas por las heurísticas greedy.